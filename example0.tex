\documentclass{article}
\usepackage{graphicx}
%\usepackage[xetex]{graphicx}
%\usepackage{fontspec,xunicode}
%\defaultfontfeatures{Mapping=tex-text,Scale=MatchLowercase}
%\setmainfont[Scale=.95]{OpenDyslexic}
\usepackage{boustrophedon}

\begin{document}

\section*{boustrophedon}
\begin{boustrophedon}
Some floating-point hardware only supports the simplest operations -- addition, subtraction, and multiplication. But even the most complex floating-point hardware has a finite number of operations it can support—for example, none of them directly support arbitrary-precision arithmetic. When a CPU is executing a program that calls for a floating-point operation not directly supported by the hardware, the CPU uses a series of simpler floating-point operations. In systems without any floating-point hardware, the CPU emulates it using a series of simpler fixed-point arithmetic operations that run on the integer arithmetic logic unit. The software that lists the necessary series of operations to emulate floating-point operations is often packaged in a floating-point library.
\end{boustrophedon}

\section*{rongorongo}
\begin{rongorongo}
Some floating-point hardware only supports the simplest operations -- addition, subtraction, and multiplication. But even the most complex floating-point hardware has a finite number of operations it can support—for example, none of them directly support arbitrary-precision arithmetic. When a CPU is executing a program that calls for a floating-point operation not directly supported by the hardware, the CPU uses a series of simpler floating-point operations. In systems without any floating-point hardware, the CPU emulates it using a series of simpler fixed-point arithmetic operations that run on the integer arithmetic logic unit. The software that lists the necessary series of operations to emulate floating-point operations is often packaged in a floating-point library.
\end{rongorongo}

\section*{left-to-right}
Some floating-point hardware only supports the simplest operations -- addition, subtraction, and multiplication. But even the most complex floating-point hardware has a finite number of operations it can support—for example, none of them directly support arbitrary-precision arithmetic. When a CPU is executing a program that calls for a floating-point operation not directly supported by the hardware, the CPU uses a series of simpler floating-point operations. In systems without any floating-point hardware, the CPU emulates it using a series of simpler fixed-point arithmetic operations that run on the integer arithmetic logic unit. The software that lists the necessary series of operations to emulate floating-point operations is often packaged in a floating-point library.

\end{document}
